\documentclass[letterpaper]{article}

\usepackage{hyperref}
\usepackage{geometry}
\usepackage{etaremune}
%\usepackage{eurofont}
\usepackage{verbatim}
\usepackage{multicol}										% columns env.
	\setlength{\multicolsep}{0pt}
\usepackage{graphicx}	
\usepackage{paralist}										% compact lists
\usepackage{tikz}
% Comment the following lines to use the default Computer Modern font
% instead of the Palatino font provided by the mathpazo package.
% Remove the 'osf' bit if you don't like the old style figures.
\usepackage[T1]{fontenc}
\usepackage[sc,osf]{mathpazo}
\usepackage{setspace}
% Set your name here
\def\name{Arun Durvasula}

% Replace this with a link to your CV if you like, or set it empty
% (as in \def\footerlink{}) to remove the link in the footer:
\def\footerlink{https://github.com/arundurvasula/CV/blob/master/CV.pdf?raw=true}

%Add in-line comments
\newcommand{\ignore}[1]{}

% The following metadata will show up in the PDF properties
\hypersetup{
  colorlinks = true,
  urlcolor = black,
  pdfauthor = {\name},
  pdfkeywords = {bioinformatics, population genetics, computer science},
  pdftitle = {\name: Curriculum Vitae},
  pdfsubject = {Curriculum Vitae},
  pdfpagemode = UseNone
}

\geometry{
  body={6.5in, 9.5in},
  left=0.8in,
  top=0.5in
}

% Customize page headers
\pagestyle{myheadings}
\markright{\name}
\thispagestyle{empty}

% Custom section fonts
\usepackage{sectsty}
\sectionfont{\rmfamily\mdseries\Large}
\subsectionfont{\rmfamily\mdseries\itshape\large}

% Other possible font commands include:
% \ttfamily for teletype,
% \sffamily for sans serif,
% \bfseries for bold,
% \scshape for small caps,
% \normalsize, \large, \Large, \LARGE sizes.

% Don't indent paragraphs.
\setlength\parindent{0em}

% Make lists without bullets
\renewenvironment{itemize}{
  \begin{list}{}{
    \setlength{\leftmargin}{1em}
  }
}{
  \end{list}
}

\begin{document}

% Place name at left
{\huge \name}
\newline
 \href{https://www.twitter.com/arundurvasula}{@arundurvasula}

% Alternatively, print name centered and bold:
%\centerline{\huge \bf \name}

\vspace{0.05in}

%ADDRESS
\begin{minipage}{0.55\linewidth}
  \href{http://www.plantsciences.ucdavis.edu/plantsciences/}Department of Plant Sciences\\
  \href{http://www.ucdavis.edu/}{University of California} \\
  1 Shields Ave.\\
  Davis, CA 95616
\end{minipage}
\begin{minipage}{0.35\linewidth}
  \begin{tabular}{ll}
    Phone: & (408) 656-6358 \\
    Email: & \href{mailto:adurvasula@ucdavis.edu}{adurvasula@ucdavis.edu} \\
    Blog: & \href{http://www.arundurvasula.wordpress.com/}{www.arundurvasula.wordpress.com} \\
    Github: & \href{https://github.com/arundurvasula/}{www.github.com/arundurvasula} \\
  \end{tabular}
\end{minipage}

%EDUCATION
\section*{Education}
\begin{itemize}
 \item BS Biotechnology, Microbiology and Fermentation, University of California Davis 2015 (expected)
% \item High School Diploma, Saint Francis High School, Mountain View, CA 2011
\end{itemize}

%EMPLOYMENT
\section*{Experience and Employment}
\begin{itemize}
%\item Mobile Game Programmer, Self-Employed/Hobby. \hfill March 2014 - Present
%	\begin{itemize}
%	\item \textbf{Avalanche Escape:} sole programmer and designer for mobile (Android and iOS) game, \textit{Avalanche Escape} (not yet released). 
%	\end{itemize}
\item Research Intern, Hancock Lab. Max F. Perutz Labratories (planned). \hfill June 2015 - August 2015
\item Research Intern, Ross-Ibarra Lab. University of California, Davis. \hfill June 2013 - Present
%	\begin{itemize}
%	\item \textbf{angsd-wrapper:} created a software package that simplifies population genetic analysis using ANGSD and streamlines graphical analysis using R and Shiny.
%	\item \textbf{Re-estimation of the maize domestication bottleneck:} used genetic simulations on a large computer cluster to re-estimate the strength of the maize domestication bottleneck using R and open source scientific software.
%	\item \textbf{Lab documentation:} wrote documentation and tutorials for computer cluster usage for use by lab. 
%	\end{itemize}
\item Bioinformatician, Rowhani Lab. University of California, Davis. \hfill June 2014 - Present
%	\begin{itemize}
%	\item \textbf{Bioinformatics:} supported research and discovery of novel viruses in diseased crop plants such as \textit{Vitis vinifera}, \textit{Pistacia vera}, and \textit{Citrus sinensis}. Additionally, developed scripts and pipelines published on Github for the automated assembly, annotation, and aggregation of sequence information from high-throughput sequencing machines.
%	\item \textbf{Writing:} prepared reports summarizing methods and results for growers.
%	\end{itemize}
\item Technical Reviewer for \textit{Bioinformatics Data Skills}, O'Reilly Media. \hfill December 2013 - March 2014
%	\begin{itemize}
%	\item \textbf{\textit{Bioinformatics Data Skills} by Vince Buffalo:} ensured accuracy of code and information in upcoming book.
%	\end{itemize}
%\item Principal Programmer, Seqcoverage. \hfill July 2014
%	\begin{itemize}
%	\item \textbf{Seqcoverage:} created an alignment web service that uses bedtools, BWA, and R to examine read coverage given a reference sequence and a set of reads (open source and available on Github).
%	\end{itemize}
%\item Research Assistant, Tagkopulous Lab. University of California, Davis. \hfill January 2013 - June 2013
%	\begin{itemize}
%	\item \textbf{\textit{Escherichia coli} simulator:} created the methodology used to model the metabolism in a novel whole cell simulation of \textit{E. coli} using Matlab.
%	\end{itemize}
%\item Intern, Williams Lab. University of California, Davis, Department of Entomology. \hfill July - September 2012
%	\begin{itemize}
%	\item \textbf{Resource competition between \textit{Apis mellifera} and \textit{Peponapis pruinosa:}} field and lab work to measure pollen and nectar amounts in squash flowers after visitation by competing species of bees.
%	\item \textbf{Effects of tillage depth on \textit{Peponapis pruinosa} overwintering survival:} set up cages to house squash bees and monitored conditions by checking for nests in the ground, conducting floral surveys and checking status of squash bees released into cages.
%	\end{itemize}

\end{itemize}


%SKILLS
\section*{Skills}
\begin{itemize}
\item Programming: Python, R, Bash, Awk, C 
% 	\begin{multicols}{3}
% 	\begin{compactitem}
% 	\item R
% 	\item Python
% 	\item Bash
% 	\item Awk
% 	\item C
% 	\item Java
% 	\end{compactitem}
% 	\end{multicols}
\item Libraries: ggplot2, dplyr, shiny, bioconductor, scikit-learn, scipy
% 	\begin{multicols}{3}
% 	\begin{compactitem}
% 	\item ggplot2
% 	\item dplyr
% 	\item shiny
% 	\item bioconductor
% 	\item scikit-learn
% 	\item scipy
% 	\end{compactitem}
% 	\end{multicols}
\item Bioinformatics: BWA, SAMtools, Plink, assembly, alignment, statistical genetics, BLAST
% 	\begin{multicols}{3}
% 	\begin{compactitem}
% 	\item assembly
% 	\item alignment
% 	\item population genetics analysis
% 	\item BLAST
% 	\end{compactitem}
% 	\end{multicols}
%	\begin{itemize}
%	\item Assembly: Velvet, IDBA\_UD, CLCBio
%	\item Alignment: BWA, bowtie
%	\item Other: samtools, bedtools, Bioconductor packages, SFS\_CODE, SLiM, ANGSD
%	\end{itemize} 
\item Tools: Unix, git, slurm, sun grid engine, SQL, NoSQL, Amazon Web Services
% 	\begin{multicols}{3}
% 	\begin{compactitem}
% 	\item unix
% 	\item git
% 	\item slurm
% 	\item sun grid engine
% 	\item databases: SQL and NoSQL
% 	\item Amazon Web Services
% 	\end{compactitem}
% 	\end{multicols}
%\item Hardware: Arduino, Raspberry Pi
% \item Molecular biology: 
% 	\begin{multicols}{3}
% 	\begin{compactitem}
% 	\item PCR
% 	\item Gel electrophoresis
% 	\item RFLP
% 	\item Bacterial transformation
% 	\item Genomic DNA isolation
% 	\end{compactitem}
% 	\end{multicols}
\end{itemize}

%PUBLICATIONS
\section*{Publications}
\begin{itemize}
\item Tyler Kent, Siddartha Bhadra-Lobo, {\bf Arun Durvasula}, Jinliang Lang, Eric Fuchs, Jeffrey Ross-Ibarra. Population genomic assessment of crop-wild gene flow in the endangered wild rice \emph{Oryza glumaepatula} (2015). In preparation.
\item {\bf Arun Durvasula}, Tyler Kent, Jeffrey Ross-Ibarra. ANGSD-wrapper: scripts to streamline and visualize NGS population genetics analysis (2015). In preparation. %\url{https://github.com/arundurvasula/angsd-wrapper-paper}.
\item Timothy Beissinger, Li Wang,  {\bf Arun Durvasula}, Kate Crosby, Matthew Hufford, Jeffrey Ross-Ibarra. Patterns of Demography and Selection Since Maize Domestication (2015). In preparation.
\end{itemize}

%TEACHING EXPERIENCE
\section*{Teaching}
\begin{itemize}
\item Teaching assistant: ECL 298, Ecological Genomics (Graduate), Winter 2015
\end{itemize}

%AWARDS
\section*{Awards}
\begin{itemize}
\item Undergraduate Travel Award, UC Davis Plant Sciences, 2015
\item Vienna Biocenter Summer Internship Scholarship, 2015
\end{itemize}

%POSTER/PRESENTATIONS
\section*{Presentations and Posters}
\begin{itemize}
\item ANGSD-wrapper: scripts to streamline and visualize NGS population genetics analysis, Poster at Society for Molecular Biology and Evolution Conf, Vienna, Austria, 2015
\item Description and detection of a novel Reovirus species in Cabernet grapevines in California, Poster at American Phytopathological Society, 2015.
%\item Patterns of Demography and Selection Since Maize Domestication, Poster at Maize Genetics Conf, 2015
\end{itemize}

%COURSEWORK

\section*{References}
\begin{itemize}
\item {\bf Jeffrey Ross-Ibarra}
\begin{itemize}
\item Associate Professor
\item Dept. of Plant Science
\item University of California
\item Davis, CA. 95616
\item email: rossibarra@ucdavis.edu 
\end{itemize}
\item {\bf Maher Al Rwahnih}
\begin{itemize}
\item Project Scientist
\item Foundation Plant Services
\item University of California
\item Davis, CA. 95616
\item email: malrwahnih@ucdavis.edu 
\end{itemize}
%\item {\bf Ilias Tagkopulous}
%\begin{itemize}
%\item Assistant Professor
%\item Dept. of Computer Science
%\item University of California
%\item Davis, CA. 95616
%\item email: itagkopoulos@ucdavis.edu 
%\end{itemize}
%\item {\bf Javier Carrera}
%\begin{itemize}
%\item Postdoctoral Research Associate
%\item Dept. of Bioengineering
%\item Stanford University
%\item Stanford, CA. 94305
%\item email: jcarrera@stanford.edu
%\end{itemize}
\end{itemize}

% Footer
%\begin{center}
%  \begin{footnotesize}
%    Last updated: \today \\
%    \href{\footerlink}{\texttt{\footerlink}}
%  \end{footnotesize}
%\end{center}

\end{document}